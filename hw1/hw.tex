\documentclass[12pt,fleqn,a4paper]{article}

\usepackage[russian]{babel}
\usepackage[utf8]{inputenc}
\usepackage{amsmath}
\usepackage{amsfonts}
\usepackage{enumitem}
\usepackage{ntheorem}
\usepackage{tikz}
\usepackage{verbatim}
\usepackage{graphicx}
\usepackage{pstricks}
\usepackage{dsfont}


\usetikzlibrary{positioning}
\tikzset{main node/.style={circle,fill=blue!20,draw,minimum size=1cm,inner sep=0pt},
}


\sloppy

%\usetikzlibrary{arrows,shapes}

\tikzstyle{vertex}=[circle,fill=black,minimum size=3pt,inner sep=0pt]
\tikzstyle{edge} = [draw,thick,-]

\newtheorem{definition}{Определение}
\newtheorem*{solution}{Решение}
\newtheorem*{answer}{Ответ}

\newenvironment{task}[2] {
	\noindent\fbox{\bf {#1} {#2}.}
}{
}

\newcommand{\mytitle}[2] {
	\begin{center}
		\bf {#1} {#2}.
	\end{center}
}

\let\origenumerate\enumerate
\let\origendenumerate\endenumerate
\renewenvironment{enumerate}{\origenumerate[topsep = 0pt, noitemsep]}{\origendenumerate}

\begin{document}
	\mytitle{Домашняя работа 1.}{Оценка погрешностей}
	
	\begin{task}{}{1}
	\end{task}
	\begin{solution}
	Хотим, чтобы остаток ряда был не больше $\epsilon = 10^{-6}$.
	
	Оценим остаток ряда через интеграл (будем считать, что $n$ достаточно большое, и $n^2 > n + z$).
	
	$$\sum_{n}^{\infty} \frac{1}{k^2 - k - z} \le \int_{n}^{\infty} \frac{dx}{x^2 - x - z}$$
	
	$$\frac{1}{x^2 - x - z} = (\frac{1}{x - x_2} - \frac{1}{x - x_1}) \cdot A,$$ где $x_{1, 2} = \frac{1 \pm \sqrt{1 + 4z}}{2}, \,\,  A = \frac{1}{\sqrt{1 + 4z}}$.

    $$ \int_{n}^{\infty} \frac{dx}{x^2 - x - z} = A( \ln (x - x_2) - \ln(x - x_1)) \bigg{|}_{n}^{\infty} = A \ln \frac{x - x_2}{x - x_1} \bigg{|}_{n}^{\infty} = A \ln \frac{n - x_1}{n - x_2}$$
    
    Нам надо подобрать такое $n$, чтобы это выражение было меньше $\epsilon$. Для того, чтобы оценить это $n$ точно, можно использовать бинпоиск.
    
    
    Теперь мы хотим посчитать с помощью ускорения. Возьмём ряд $\frac{1}{k(k - 1)}$ для $k \ge 2$. Его сумма $1$.
    
    Если рассмотреть разность рядов, то получим $\sum_{1}^{\infty} \frac{1}{k^2 - k - z} = - \frac{1}{z} + \sum_{2}^{\infty} \frac{1}{k^2 - k - z} = - \frac{1}{z} + \sum_{2}^{\infty} (\frac{1}{k^2 - k - z} - \frac{1}{k^2 - k}) + 1 = 1 - \frac{1}{z} + z \sum_{2}^{\infty}  (\frac{1}{(k^2 - k - z)k (k - 1)})$
    
    Теперь нам снова надо оценить интеграл $\frac{1}{(k^2 - k - z)k (k - 1)}$. Скажем, что если мы уменьшим знаменатель, то интеграл только увеличится, и оценим его, как $\frac{1}{(k - 1)^4}$ (выберем такое $z$, чтобы эта оценка выполнялась). $$\int_{n}^{\infty} \frac{dx}{(x - 1)^4} = -\frac{1}{3(x - 1)^3}\bigg{|}_{n}^{\infty} = \frac{1}{3(n - 1)^3}.$$
    
    В таком случае, $\frac{z}{3 (n - 1)^ 3} \approx \epsilon$, то есть $n \approx \sqrt[3]{\frac{z}{3 \epsilon}} + 1.$
    
    (Первую часть можно сделать также: оценить каждое слагаемое, начиная с какого-то места, как $\frac{1}{(k - 1)^2}$, посчитать интеграл, начиная с $n$, он будет равен $\frac{1}{(n - 1)}$, а дальше $\frac{1}{(n - 1)} =  \epsilon$, то есть $n \approx \frac{1}{\epsilon} + 1$. Так как это работает в разы быстрее, то в коде сделано так).
    
	\end{solution}

	\begin{task}{}{2}
\end{task}
\begin{solution}
	$$s(z) = \sum_{k = 1}^{\infty} \frac{1}{2k - 1} z^{k}$$
	
	Посмотрим на отношение соседних элементов этого ряда. $$\frac{z^{k + 1} / (2 k + 1)}{z^k / (2k -1)} \rightarrow z$$
\end{solution}
	
	
	
	
\end{document}